%%
%% This is file `sample-acmsmall.tex',
%% generated with the docstrip utility.
%%
%% The original source files were:
%%
%% samples.dtx  (with options: `acmsmall')
%% 
%% IMPORTANT NOTICE:
%% 
%% For the copyright see the source file.
%% 
%% Any modified versions of this file must be renamed
%% with new filenames distinct from sample-acmsmall.tex.
%% 
%% For distribution of the original source see the terms
%% for copying and modification in the file samples.dtx.
%% 
%% This generated file may be distributed as long as the
%% original source files, as listed above, are part of the
%% same distribution. (Tdrhe sources need not necessarily be
%% in the same archive or directory.)
%%
%% The first command in your LaTeX source must be the \documentclass command.
\documentclass[acmsmall,nonacm]{acmart}
%% NOTE that a single column version is required for 
%% submission and peer review. This can be done by changing
%% the \doucmentclass[...]{acmart} in this template to 
%% \documentclass[manuscript,screen]{acmart}
%% 
\usepackage{tcolorbox}
\usepackage{moresize}

\newcommand{\M}{\textbf{M}}
\newcommand{\D}{\textbf{D}}

%% To ensure 100% compatibility, please check the white list of
%% approved LaTeX packages to be used with the Master Article Template at
%% https://www.acm.org/publications/taps/whitelist-of-latex-packages 
%% before creating your document. The white list page provides 
%% information on how to submit additional LaTeX packages for 
%% review and adoption.
%% Fonts used in the template cannot be substituted; margin 
%% adjustments are not allowed.
%%
%% \BibTeX command to typeset BibTeX logo in the docs
\AtBeginDocument{%
  \providecommand\BibTeX{{%
    \normalfont B\kern-0.5em{\scshape i\kern-0.25em b}\kern-0.8em\TeX}}}
\newcommand{\etal}{\textit{et al.}}

%% Rights management information.  This information is sent to you
%% when you complete the rights form.  These commands have SAMPLE
%% values in them; it is your responsibility as an author to replace
%% the commands and values with those provided to you when you
%% complete the rights form.
\setcopyright{acmcopyright}
\copyrightyear{2018}
\acmYear{2018}
\acmDOI{10.1145/1122445.1122456}


%%
%% These commands are for a JOURNAL article.
\acmJournal{JACM}
\acmVolume{37}
\acmNumber{4}
% \acmArticle{111}
\acmMonth{8}

%%
%% Submission ID.
%% Use this when submitting an article to a sponsored event. You'll
%% receive a unique submission ID from the organizers
%% of the event, and this ID should be used as the parameter to this command.
%%\acmSubmissionID{123-A56-BU3}

%%
%% The majority of ACM publications use numbered citations and
%% references.  The command \citestyle{authoryear} switches to the
%% "author year" style.
%%
%% If you are preparing content for an event
%% sponsored by ACM SIGGRAPH, you must use the "author year" style of
%% citations and references.
%% Uncommenting
%% the next command will enable that style.
%%\citestyle{acmauthoryear}

%%
%% end of the preamble, start of the body of the document source.
\begin{document}

%%
%% The "title" command has an optional parameter,
%% allowing the author to define a "short title" to be used in page headers.
\title{Real Roots Isolation of Polynomials}

%%
%% The "author" command and its associated commands are used to define
%% the authors and their affiliations.
%% Of note is the shared affiliation of the first two authors, and the
%% "authornote" and "authornotemark" commands
%% used to denote shared contribution to the research.
\author{Xinlong Yi(862188160)}
\email{xyi007@ucr.edu}
\affiliation{%
  \institution{\\Computer Science and Engineering Department, University of California Riverside}
  \streetaddress{900 University Ave}
  \city{Riverside}
  \state{California}
  \country{USA}
  \postcode{92507}
}


%%
%% By default, the full list of authors will be used in the page
%% headers. Often, this list is too long, and will overlap
%% other information printed in the page headers. This command allows
%% the author to define a more concise list
%% of authors' names for this purpose.
% \renewcommand{\shortauthors}{Trovato and Tobin, et al.}

%%
%% The abstract is a short summary of the work to be presented in the
%% article.

%%
%% The code below is generated by the tool at http://dl.acm.org/ccs.cfm.
%% Please copy and paste the code instead of the example below.
%%
% \begin{CCSXML}
% <ccs2012>
%  <concept>
%   <concept_id>10010520.10010553.10010562</concept_id>
%   <concept_desc>Computer systems organization~Embedded systems</concept_desc>
%   <concept_significance>500</concept_significance>
%  </concept>
%  <concept>
%   <concept_id>10010520.10010575.10010755</concept_id>
%   <concept_desc>Computer systems organization~Redundancy</concept_desc>
%   <concept_significance>300</concept_significance>
%  </concept>
%  <concept>
%   <concept_id>10010520.10010553.10010554</concept_id>
%   <concept_desc>Computer systems organization~Robotics</concept_desc>
%   <concept_significance>100</concept_significance>
%  </concept>
%  <concept>
%   <concept_id>10003033.10003083.10003095</concept_id>
%   <concept_desc>Networks~Network reliability</concept_desc>
%   <concept_significance>100</concept_significance>
%  </concept>
% </ccs2012>
% \end{CCSXML}

% \ccsdesc[500]{Computer systems organization~Embedded systems}
% \ccsdesc[300]{Computer systems organization~Redundancy}
% \ccsdesc{Computer systems organization~Robotics}
% \ccsdesc[100]{Networks~Network reliability}

%%
%% Keywords. The author(s) should pick words that accurately describe
%% the work being presented. Separate the keywords with commas.
% \keywords{datasets, neural networks, gaze detection, text tagging}


%%
%% This command processes the author and affiliation and title
%% information and builds the first part of the formatted document.
\maketitle
% Presentation link: \url{https://docs.google.com/presentation/d/1M79AJubJ6nv7tiNiWuUS_wnHdhcYuvaeLgdKnoC8FTw/edit?usp=sharing}

% \newpage

\section{Abstract}

Computing the real roots of univariate polynomial is one of fundamental tasks in
numeric and computer algebra. Usual root-finding algorithms for computing the
real roots of a polynomial may produce some real roots, but cannot generally
certify having found all real roots.

In this project, we implemented a basic real roots isolation program based on
Budan's Theorem and Continued Fraction Method. Both methods are developed from 
Descartes' rule of signs. Then we compare the running time of these two methods
and get the conclusion that TODO.


\section{Problem}
\begin{tcolorbox}
\textbf{Given} a model $\M$, \\
\textbf{Train} $\M$ on a BGP anomaly dataset $\D$ \\
\textbf{Detect} BGP anomalies in other dataset(s)
\end{tcolorbox}

As with any machine learning literature, the current literature for BGP anomaly detection does test for generalization, but only between training and testing sets \textit{drawn from the same anomaly}. While these sorts of results are indicative of good performance for the tested anomalies, they cannot tell us much about the ability of models to generalize across multiple anomalies.

In this project, we present a novel analysis of BGP anomaly detection methods \textit{between} different anomalies. We compare the generalization performance of several anomaly detection approaches, some of which have never been applied to this domain before. We measure the generalizability of these models using accuracy and F1 scores on test sets consisting of multiple anomalies. Furthermore, we explore the features responsible for generalization performance and attempt to interpret their importance.
% Future anomalous events may appear radically different from the events used to piece together previous models, meaning that models trained using supervised methods can offer little guarantee of detecting future events, no matter how accurate they may be at detecting the labeled events. 

% This is essentially an issue of generalizability between different anomalous events, which can be approached using a number of different methods. For this project, we will apply two general approaches commonly used in anomaly detection: one-class learning and unsupervised methods. 
% \begin{itemize}
%     \item \textbf{Regularized classification} - We will attempt to learn a model on one set of anomalous event data and apply well-documented regularization methods to generalize the model to one or more sets of unseen anomalous data.
%     \item \textbf{One-class learning} - The model will be trained on normal BGP traffic data and we will attempt to use it to discriminate between normal and anomalous events.
% \end{itemize}
% \begin{tabbing}
% \textbf{Input:} \hspace{4.2em} \= Time series of BGP features\\
% \textbf{Desired Output:} \> Low generalization error between training and testing events\\
% \textbf{Challenges:} \> Datasets representative of (differences between) events, balancing ability to fit \\
% \> and ability to generalize, dataset imbalance (between events)
% \end{tabbing}

\subsection{Scope}

An "anomaly" in networking can be difficult to define, as it is difficult to precisely characterize the appearance of "normal" traffic. A 2017 survey of BGP anomaly detection approaches by Al-Musawi et al. \cite{al2016bgp} constructs a taxonomy based on the cause of the anomalous behavior. E.g. "Direct" anomalies are caused by problems with the network itself, such as prefix hijacks ("direct intended") or origin misconfigurations("direct unintended"), while indirect anomalies occur as a result of events such as a worm spreading across the Web. 

Finding data for many different anomalies was difficult itself, but compounding that problem was the fact that datasets generally share very few of the same features, which makes comparisons between them difficult. As a result, we limit our dataset to three indirect anomalies caused by computer worms in the early 2000s: Code Red I, Slammer, and Nimda. It would be ideal to include different types of anomalies from different time periods, but we cannot both gather the necessary data and perform the analysis with the time we have. The relative similarity of these anomalies can be beneficial though, as any deficiencies in generalization will represent a sort of upper bound on generalization performance. 

\subsection{Previous and Related Work}

Thus far, machine learning for BGP anomaly detection has not grown very complex, and still uses relatively simple methods such as SVMs \cite{Scholkopf1998SVMsTheory}, or simple RNNs such as LSTMs (Long Short-Term Memory) \cite{Hochreiter1997LongMemory} or GRUs (Gated Recurrent Unit) \cite{Chung2014EmpiricalModeling}. 

Other anomaly detection tasks have seen the use of more complex models involving deep learning such as, GANs (Generative Adversarial Networks) \cite{goodfellow2014generative} \cite{Ngo2019FenceDetection} and Deep SVDD (Support Vector Data Description) \cite{ocnn-ruff18a}. These have yet to be applied to the specific domain of BGP anomaly detection.

As stated above, there currently is no literature which compares the inter-anomaly generalizability of these models, so we will have to design our experimental framework from scratch. 
% BGP anomaly detection technique is a hot topic aiming at detecting and alerting anomalous events so as to minimize the damage it causes. 
% According to the survey\cite{Ahmed15}, typical types of anomalies can be concluded as point anomaly, contextual anomaly and collective anomaly. Worms, power outages, and BGP router configuration errors are all considered as anomalous events. These attacks are sharp, resulting in sustained increases in the number of announcement or withdrawal messages exchanged by BGP routers\cite{Ding18}.

% Machine learning based(classification, clustering, statistical analysis) network anomaly detection is an important category in all the detection methods, thus making both datasets and feature selection algorithms significant. Well-known datasets like Route Views, Réseaux IP Européens (RIPE), and BCNET\cite{Ding18} are widely chosen by researchers to apply various feature selection algorithms. One outstanding approach is the automatic feature extraction by neural network\cite{Xu20} for it solving the problem of manually selecting and deciding the statistical features used for anomaly detection. Generally, the output of anomaly detection techniques are scores and/or binary labels, and evaluation is generated based on both criteria.

% Neural networks have been commonly applied to anomaly detection. For supervised anomaly detection, many recent papers use LSTMs (Long Short-Term Memory) \cite{Hochreiter1997LongMemory} and GRUs (Gated Recurrent Unit) \cite{Chung2014EmpiricalModeling}, which work on a timeseries, rather than an individual sample. Generative Adversarial Networks (GANs) \cite{goodfellow2014generative} were original developed for image generation, but have since been applied to a wider variety of tasks, including anomaly detection. Fence-GAN \cite{Ngo2019FenceDetection}

% For one-class learning, most models use the idea of fitting a hyperplane or hypersphere to the data and then using that to classify anything outside of it as anomalous. One common model for this purpose is the Support Vector Data Description (SVDD) \cite{Tax2004} model, which is insprired by the Support Vector Machine (SVM) \cite{Scholkopf1998SVMsTheory}. Another relatively common family of models for this task are the one-class neural networks (OC-NN) \cite{OCNN}. One example of this is the Deep SVDD \cite{ocnn-ruff18a}, which has been successfully applied to image classification.

% However, most of these one-class techniques have yet to be applied specifically to BGP anomaly detection. The closest work we were able to find was by Allahdadi \etal \cite{Allahdadi2017ADetection}, which uses a one-class SVM.

\section{Solution}
Since previous work in BGP anomaly detection has not thus far examined the generalization performance between different anomalies, we begin by surveying the generalizability of different anomaly detection methods on our three separate worm datasets. Using the information gleaned from this survey, we select the method with the highest generalizability (which we define as the highest mean F1 score), and perform a feature ablation experiment with it in order to ascertain which networking features are most important for generalizing between different the three different worms. 


\subsection{Generalization Performance}
In order to evaluate the inter-anomaly generalization performance of each method, we use three combinations of training and testing data where each method is trained on one of the anomalies and tested on the other two. 
% \begin{center}
% \begin{tabular}{|c|c|c|c|c|c|c|c|c|}
%     \hline
%     \textbf{Method} & \textbf{C Acc} & \textbf{C F1} & \textbf{N Acc} & \textbf{N F1} & \textbf{S Acc} & \textbf{S F1} & \textbf{Mean Acc} & \textbf{Mean F1}\\
%     \hline
%     Jiaojiao Cheng & 70\%/50\% & 25\% & 25\% & 25\% & 95\% \\
%     Colin Lee & \% & 25\% & 25\% & 25\% & 95\%\\
%     Zhuocheng Shang & 85\%/50\% & 25\% & 25\% & 25\% & 90\%\\
%     William Shiao & 90\%/50\% & 25\% & 25\% & 25\% & 90\% \\
%     \hline
% \end{tabular}
% \end{center}

\begin{table} [ht!]
\centering
% \begin{ssmall}
\begin{tabular}{lll|ll|ll|ll}
\toprule
    & \multicolumn{2}{c|}{\textbf{Nimda}} & \multicolumn{2}{c|}{\textbf{Code Red}} & \multicolumn{2}{c|}{\textbf{Slammer}} & \multicolumn{2}{c}{\textbf{Mean}} \\
    \midrule
    \multicolumn{1}{c}{Model} & \multicolumn{1}{c}{Acc.} & \multicolumn{1}{c|}{F1} & \multicolumn{1}{c}{Acc.} & \multicolumn{1}{c|}{F1} & \multicolumn{1}{c}{Acc.} & \multicolumn{1}{c|}{F1} & \multicolumn{1}{c}{Acc.} & \multicolumn{1}{c}{F1} \\
    \bottomrule
    \toprule
    \multicolumn{9}{c}{\textbf{One-class Methods}} \\
    \midrule
    OC-SVM &0.102  &0.185  &0.253 &0.404  &0.237  &0.384  &0.198 & 0.324 \\
    Entropy OC-SVM &0.102  &0.185  &0.253 &0.404  &0.237  &0.384  &0.198 & 0.324\\
    Autoencoder & 0.903 & 0.501 & 0.608 & 0.282 & 0.745 & 0.530 & 0.752 & \textbf{0.438} \\
    Deep SVDD & 0.921 & 0.479 & 0.804 & 0.354 & 0.726 & 0.388 & \textbf{0.817} & 0.407 \\
    \bottomrule
    \toprule
    \multicolumn{9}{c}{\textbf{Unsupervised Methods}} \\
    \midrule
    KNN & 0.895 & 0.488 & 0.786 & 0.453 & 0.778 & 0.449 & \textbf{0.820} & \textbf{0.463}\\
    Isolation Forest  & 0.856 & 0.395 & 0.769 & 0.355 & 0.769 & 0.324 & 0.798 & 0.358 \\
    PCA-based & 0.827 & 0.295 & 0.711 & 0.245 & 0.747 & 0.177 & 0.762 & 0.239 \\
    LOF w/ Feature Bagging & 0.779 & 0.249 & 0.679 & 0.178 & 0.695 & 0.176 & 0.717 & 0.201 \\
    Angle-based Outlier Detector& 0.892 & 0.488 & 0.768 & 0.372 & 0.772 & 0.382 & 0.811 & 0.414\\
    \bottomrule
\end{tabular}
% \end{ssmall}
\caption{\label{tab:results} All of the scores shown are the scores when the model is trained on the listed dataset and evaluated on the remaining two datasets. LOF stands for Local Outlier Factor.}
\end{table}

We note that autoencoders appear to be the best one-class method and KNN is the best unsupervised method for generalizing between these datasets. In general, the unsupervised methods outperform the one-class methods, which is an expected result, given that the unsupervised methods learn from both anomalous and unanomalous data, whereas the one-class methods are limited to learning their decision boundaries from the unanomalous data. 


\subsection{Feature Ablation Analysis}
In order to explore \textit{what} makes these models generalizable, we ablatively remove features from our data and note which features are most important for the best methods. We then note the most and least important features based on how much they decrease the F1 score of the methods. 

\begin{table} [ht!]
\centering
\begin{tabular}{c|c|c}
\toprule
     \textbf{Method} & \textbf{Most important} & \textbf{Least important} \\
     \hline
     KNN & Number of withdrawn NRLI prefixes & Packet size \\
     & Number of announcements  & Number of duplicate withdrawals \\
     \hline
     ABOD & Number of withdrawn NRLI prefixes & Packet size\\
     & Avg unique AS-path & Number of duplicate withdrawals \\
     \hline
     AE & Avg AS-path length & Max AS-path length = 8 \\
     & Max AS-path length & Number of duplicate withdrawals \\
\bottomrule
\end{tabular}
\caption{\label{tab:feats} Top two most important and least important features for the three best methods: K-nearest neighbor, Angle-based Outlier Detector, Autoencoder}
\end{table}

In general, it seems that the number of withdrawn NLRI prefixes and one of the AS-path length features are important to the generalization performance of the each of the models. This suggests that generalizable models must key in on features that are informative about the reachability of other ASs. Worms such as the ones investigated in these datasets typically affect networks most when attempt to rapidly replicate themselves, overloading the capacity of the network and leading to Denial of Service events. As such, it would follow that some nodes become unreachable and different AS-paths must be found. 

Least important features included packet size, and the number of duplicate withdrawals. Packet size is negligible as packet size can vary normally for any number of reasons. That the number of duplicate withdrawals is unimportant to all three suggests that normal traffic sees a similar number of duplicate withdrawals as anomalous traffic.

\section{Challenges}

%We had trouble finding recent datasets for BGP anomaly detection, so we had to use these three datasets because they were one of the few available 

Since nobody else has performed this type of analysis on BGP data before, we also had difficulty deciding how exactly to evaluate our methods. We had a hard time deciding on how to evaluate the generalizabilty of our methods.

Ideally, the anomalies in our dataset would have represented a diverse array of anomalies from different time periods, but almost every dataset we encountered used a different set of features, which would have made comparison and analysis impossible with our current experimental framework.

We had to write code to load the datasets properly because of how we chose to evaluate the data. We also to re-implement several of methods because there were no implementations online that worked with our data. Examples of these were the Fence-GAN \cite{Ngo2019FenceDetection}, which had old code online, but it only worked with images and no longer ran on newer hardware. We reimplemented this from scratch only to find that it would take too long to train to a reasonable level of accuracy (which is why it is not included in the table). The autoencoder method was another example of this, where the papers describing it for anomaly detection lacked detail and a sample implementation, so we had to guess about how to fill in some of the blanks. Another challenge is the low accuracy and F1-score performance by the OC-SVM and entropy OCSVM, it lead our team to consider which entropy or one class method would fit better on such type of BGP dataset.



% \section{Contributions}

%We propose to develop machine learning models to classify imbalanced data and anomalies from BGP traffic dataset. \cite{Xianbo19} \cite{Wu19} We also expected to compare performance between different models over varied dataset.

%We plan to benchmark our approach against the SVM approach by Allahdadi \etal \cite{Allahdadi2017ADetection} and the GRU and LSTM approaches by Li \etal \cite{li2019machine}. 

Our planned contributions to this body of knowledge are as follows:

\begin{enumerate}
    %\item Analyze dataset to manage feature engineering which helps to detect imbalanced data and anomalies.
    \item Re-implement both semi-supervised and fully supervised classification models to detect anomalies.
    \item Evaluate the generalization error of the SVM approach by Allahdadi \etal \cite{Allahdadi2017ADetection}, and the GRU and LSTM approaches by Li \etal \cite{li2019machine} on different datasets.
    \item Develop and implement a novel model for anomaly detection that has improved generalizability between datasets.
    \item Possibly perform feature engineering to improve generalizability of models between different datasets.
    % \item Compare detection results with referenced papers applied similar models.
    % \item Validate models with modern data or different distributed data.
\end{enumerate}






%%
%% The acknowledgments section is defined using the "acks" environment
%% (and NOT an unnumbered section). This ensures the proper
%% identification of the section in the article metadata, and the
%% consistent spelling of the heading.
% \begin{acks}
% % To Robert, for the bagels and explaining CMYK and color spaces.
% \end{acks}

%%
%% The next two lines define the bibliography style to be used, and
%% the bibliography file.
\bibliographystyle{ACM-Reference-Format}
\bibliography{sample-base,references}


%%
%% If your work has an appendix, this is the place to put it.
\appendix


\end{document}
\endinput
%%
%% End of file `sample-acmsmall.tex'.
