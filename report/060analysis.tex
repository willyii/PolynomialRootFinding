\section{Results and analysis} \label{analysis}

\subsection{Running Time}

With above implementation and operation, we test the running time of these two
methods. Results can be found in table~\ref{tb2}. Since we do not pay attention
to the polynomials with degree more than 6, we only test the polynomials with
degree 6 in this section. These test polynomials are generated with random
coefficients from $-c$ to $c$. 

\begin{center}
\label{tb2}
\begin{tabular}{ |c|c|c| } 
 \hline

 $c$  & Budan's Theorem & Continued Fraction\\ 

 \hline
 10   & 195 $us$  & 26 $us$\\ 
 20   & 240 $us$  & 22 $us$\\ 
 50   & 243 $us$  & 32 $us$\\ 
 100   & 257 $us$  & 32 $us$\\ 
 1000   & 259 $us$  & 32 $us$\\ 
 \hline
\end{tabular}
\end{center}

From table~\ref{tb2}, we can see that Budan's theorem method takes longer than
the continued fraction method. This might be because in Budan’s theorem, we need
to get the upper bound of root $up\_b$ first and get $P(x+up\_b)$ to check how
many positive real roots. However in continued fraction, sign variance of $P$
can tell us the information about positive roots directly. 

Furthermore, in Budan's theorem method, sometimes the search range needs to be
small enough to figure out whether there are roots or not. Take $P(x) = x^4 +1 $
as an example. In Budan's theorem method with search range $[0-\epsilon,
0+\epsilon]$, $v_{0+\epsilon}(P) - v_{0-\epsilon}(P) = 4$ is always valid, no
matter how small the $\epsilon$ is. Therefore, the bisection process will
terminate only when $2\epsilon < MINIMAL\_RANGE$. However, the continued
fraction method will check zero root first, then find no positive root since
$v_0(P(x)) = 0$ and no negative roots since $v_0(P(-x))=0$. That's the reason
why the Budan’s theorem method takes longer than the continued fraction method.

Above polynomials are generated with random coefficients, which means it cannot
guarantee that those polynomials have real roots. In order to figure out the
relationship between the running time with magnitude of roots, we designed
following experiments. Test polynomials have 6 roots and every root has $p =
0.5$ probability same as the previous one. Distinct roots are generated randomly
in range $[-max\_root, max\_root]$. Degree of test polynomials is 6. Running
times are shown in table~\ref{tb3}.


\begin{center}
\label{tb3}
\begin{tabular}{ |c|c|c| } 
 \hline

 $max\_root$  & Budan's Theorem & Continued Fraction\\ 

 \hline
 1   & 23 $us$  & 23 $us$\\ 
 10   & 28 $us$  & 25 $us$\\ 
 50   & 34 $us$  & 36 $us$\\ 
 100   & 36 $us$  & 51 $us$\\ 
 200   & 39 $us$  & 86 $us$\\ 
 500   & 42 $us$  & 180 $us$\\ 
 1000   & 47 $us$  & 300 $us$\\ 
 \hline
\end{tabular}
\end{center}

First, we can observe that when roots go larger the continued fraction method
takes longer than the Budan's theorem method. It is noticeable that $p'=p(x+1)$
step in Algorithm~\ref{alg4} only shifts polynomial by $1$ each time. Therefore,
when roots are large, it takes longer for the continued fraction method to shift
the polynomial to the place around roots. This place needs to be optimized in
the future.  Shifting can be flexible according to the lower bound of roots of
polynomials.

Furthermore, we can also notice that the Budan’s theorem method runs faster than the 
previous one while the continued fraction method runs slower. This mainly
because these polynomials are guaranteed to have 6 real roots. With 6 real
roots, there are no conjugate complex roots, like $x^4-1$. This condition makes
Budan’s theorem method terminate earlier. Besides that, some factors of original
polynomial might only have degree of 2 or 1. These factors can be solved very
quickly. However, for the continued fraction method, it takes longer to shift
polynomials to the place around roots.

\subsection{Error Analysis}

Although this implementation can work correctly in most cases, there are some
conditions that could lead to failure.

Since both methods need to work on polynomials with no repeat roots, square-free
decomposition becomes the most important step in the program. However, Yun's
algorithm requires exact divisions and $GCD$ success, which is hard to achieve
in floating point computation.

As mentioned before, we introduce interval arithmetic to handle the errors from
inexact computation and keep track of errors. We find that errors are related to
the magnitude of coefficients. Errors might be enlarged during the process of
$GCD$ and long division. Let's simulate the first step of $GCD(P,P')$, which is
$rem(P,P')$ where $P' = \frac{dP}{dx}$.

Given a polynomial $P=a_0+a_1*x + a_2*x^2...+a_n*x^n$, and its derivative
$P'=a_1+2*a_2*x...+n*a_n*x^{n-1}$. We assume all of coefficients of $P$ and $P'$
can be represented by computer perfectly. Which means the width of interval
$a_i$ is zero.

Then the processes to calculate $rem(P,P')$ are:

\begin{align*}
  div_1 &= 1/n \\ 
  rem_1 &= P - div_1 * x * P' =  a_0 + \frac{n-1}{n}a_1*x+ \frac{n-2}{n}a_2*x^2....
\frac{1}{n}*a_{n-1}x^{n-1}\\ 
       &= \sum_{i=0}^{n-1} \frac{n-i}{n} a_i x^i \\
  div_2 &= \frac{a_{n-1}}{a_n} * \frac{1}{n^2} \\ 
       &= div_1 * \frac{a_{n-1}}{na_n}\\
  rem_2 &= rem_1 - div_2 * P' = (a_0 - div_2*a_1) + (\frac{n-1}{n}*a_1 - div_2 *
  2*a_2)+ ... \\
      &= \sum_{i=0}^{n-2} (\frac{n-i}{n}*a_i -
      \frac{a_{n-1}*a_{i+1}}{n^2*a_n}(i+1)) x^i \\
      &= \sum_{i=0}^{n-2} ((1-div_1*i)*a_i -
       div_2 * a_{i+1}(i+1)) x^i
\end{align*} 

Since $rem_2$ only has degree $n-2$ which is one less than $P'$, it is the
answer of $rem(P,P')$. Let's assume errors appear in computation of $div_1$.
Which means $div_1 = \frac{1}{n}\pm \epsilon$. Then we set $\xi =
\frac{a_{n-1}}{na_n}$, $div_2 = div_1 * \xi = \frac{a_{n-1}}{a_nn^2}  \pm
\epsilon \xi$. The width of value range of $div_2$ might be changed by $\xi$.
It's noticeable that start from $div_1$ the coefficients of intermediate
polynomials will be expressed as a set of possible value, instead of exact
value. Therefore, every square-free factor might be written in form as $P_i =
c_0[\pm \epsilon_0]+ c_1[\pm \epsilon_1]*x + c_2[\pm \epsilon_2]*x^2+ ... +
c_k[\pm \epsilon_k]*x^k$ after square-free decomposition.

If real roots of a polynomial are close enough, a little change of the
coefficients might change the roots a lot, not only the value, but also the
nature. For example, a square-free polynomial $P = (x-3.003)*(x-3.004)=x^2  -
6.007x + 9.021012$. If we only add $0.00000025$ to the constant term, these two
roots will become repeated $3.0035$. Besides that, if we add $1e^{-7} +
0.00000025$ to constant term, there will be no real roots for this polynomial.

As mentioned before, after square-free decomposition, error of coefficients may
not be avoidable. If roots of these polynomials are not distinctive enough,
real-root isolation might fail since it is hard to figure out the property of
these close roots. This problem seems inherent to specific problems. If we do not
have extra information about polynomials, it's hard to find out close distinct
roots.
