\section{Methodology}

In this section, I will introduce the methods that applied to this project.
Since both two methods are only work on square-free polynomials, the first step
of this project is appling square free decomposition to original polynomials to
avoid repeat roots. After square free decomposition, methods based on Budan's
Theorem and Continued Fraction will be applied to square-free polynomials. They both based on
Descartes' rule of sign to check how many real roots in a interval. The continued fraction
method also uses Mobius transformation, which will be introduce in Continued Fraction
subsection.

\subsection{Square Free Decomposition} 
In mathematics, a square-free polynomial is a polynomial defined over a field
that does not have as a divisor any square of a non-constant
polynomial\cite{Yuns}. Usually, a square-free polynomial refers to the
polynomials with no repeated roots. This project applied Yun's
algorithm\cite{Yuns} to perform square-free decomposition. It's based one the
succession of Greatest Common Divisor(GCD).

\subsubsection{Greatest Common Divisor}

In algebra, the greatest common divisor of two polynomials is a polynomial, 
of the highest possible degree, that is a factor of both the two original
polynomials. This concept is simillar to the GCD of two integers. 

This project applied Euclid's algorithm to compute the GCD of two polynomials.
It's simillar to the  Euclidean division for integers. In the Algorithm~\ref{alg1}
, $rem(a,b)$ refers to the remainder of Euclidean division of
polynomial $a$ and polynomial $b$.

\begin{algorithm}[H]
\label{alg1}
\SetAlgoLined
  \SetKwInOut{Input}{input}
  \SetKwInOut{Output}{output}

  \Input{$P1$: a univariate polynomial\newline
         $P2$: a univariate polynomial}
  \Output{Greatest common divisor of $P1$ and $P2$}

  $r_0 = P1$\;
  $r_1 = P2$\;
  \For{$i=1; r_i \neq 0; i++$}{
    $r_{i+1} = rem(r_{i-1},r_i)$
  }

  \Return{$r_{i-1}$}

\caption{GCD of two polynomials}
\end{algorithm}

\subsubsection{Yun's Algorithm}

\subsection{Budan's Theorem}

\subsubsection{Descartes' rule of signs}

\subsubsection{Budan's Theorem}

\subsection{Continued Fraction}

\subsubsection{Mobius Transformation}

\subsubsection{Continued Fraction}

