\section{Methodology} \label{methods}

%This section introduce the methods that applied to this project.
%Since both two methods are only work on square-free polynomials, the first step
%of this project is applying square free decomposition to original polynomials to
%avoid repeat roots. After square free decomposition, methods based on Budan's
%Theorem and Continued Fraction will be applied. They both based on
%Descartes' rule of sign to check how many real roots in a interval. The continued fraction
%method also uses Mobius transformation, which will be introduce in Continued Fraction
%subsection.

\subsection{Square Free Decomposition} 
In mathematics, a square-free polynomial is a polynomial defined over a field
that does not have a divisor any square of a non-constant
polynomial\cite{Yuns}. Usually, a square-free polynomial refers to the
polynomials with no repeated roots. This project applied Yun's
algorithm to perform square-free decomposition. It's based one the
succession of Greatest Common Divisor(GCD).

\subsubsection{Greatest Common Divisor}

In algebra, the greatest common divisor of two polynomials is a polynomial, 
of the highest possible degree, that is a factor of both the two original
polynomials. This concept is similar to the GCD of two integers. Ideally, both
polynomials divided by their greatest common divisor should have no remainder.

This project uses Euclidean algorithm to compute the GCD of two polynomials.
In Algorithm~\ref{alg1}, $rem(a,b)$ refers to the remainder of Euclidean division of
polynomial $a$ and polynomial $b$.

\begin{algorithm}[H]
\label{alg1}
\SetAlgoLined
  \SetKwInOut{Input}{input}
  \SetKwInOut{Output}{output}

  \Input{$P1$: a univariate polynomial\newline
         $P2$: a univariate polynomial}
  \Output{Greatest common divisor of $P1$ and $P2$}

  $r_0 = P1$\;
  $r_1 = P2$\;
  \For{$i=1; r_i \neq 0; i=i+1$}{
    $r_{i+1} = rem(r_{i-1},r_i)$
  }

  \Return{$r_{i-1}$}

\caption{GCD of two polynomials}
\end{algorithm}

\subsubsection{Yun's Algorithm}

Based on the success of GCD, Yun developed a square-free decomposition
algorithm for univariate polynomials. 

Given a primitive polynomial $P$, assume $P=P_1P_2^2...P_n^n$ is desired
factorization. Yun's algorithm will compute square-free polynomials $P_i$ and
the subscript refers to the times this square-free polynomial appears. Which
means $P_i$ appears $i$ times in original polynomial. This process is described in
Algorithm~\ref{alg2} formally.

\begin{algorithm}[H]
\label{alg2}
\SetAlgoLined
  \SetKwInOut{Input}{input}
  \SetKwInOut{Output}{output}

  \Input{Primitive polynomial $P$}
  \Output{List of square-free polynomials}

  $G = GCD(P, dP/dx)$\;
  $C_1=P/G$\;
  $D_1 = (dP/dx)/G - dC_1/dx$\;

  \For{$i=1,C_i\neq 0;i=i+1$}{
    $P_i= GCD(C_i,D_i)$\;
    $C_{i+1}= C_i/P_i$\;
    $D_{i+1}= D_i/P_i - dC_{i+1}/dx$\;
  }

  \Return{$P_1, P_2...P_n$}

\caption{Yun's Square-free Decomposition Algorithm}
\end{algorithm}

\subsection{Budan's Theorem}

Budan's theorem is a theorem used for bounding the number of real roots in a
given interval. Given a univariate polynomial $P$, we denote
$\#_{l,r}(P)$ as the number of real roots of $P$ in half-open interval $(l,r]$.
Then we denote $v_h(P)$ as the number of sign changes in coefficients of
polynomial $P_h$, where $P_h(x) = P(x+h)$.

Budan's theorem states that $v_l(h) - v_r(h) - \#_{l,r}(P)$ is a nonnegative
even integer. From this statement, we can know that if $v_l(h) - v_r(h) = 1
\text{ or } 0$, there is only one or zero real root in interval $(l,r]$. 

Based on this theorem, this project combines bisection method with Budan's
theorem to isolate the real roots.  This process described in
Algorithm~\ref{alg3}.

\begin{algorithm}[H]
\label{alg3}
\SetAlgoLined
  \SetKwInOut{Input}{input}
  \SetKwInOut{Output}{output}

  \Input{A square-free polynomial $P$} 
  \Output{List of intervals contains only one real root}

  $ret = []$\;
  $up\_bound = Upper(P)$\;
  $low\_bound = - up\_bound$\;
  $search = [(low\_bound, up\_bound)]$\;

  \While{$search$ not empty}{
    $l, r = pop(search)$\;
    $vl = sign\_change(P(x+l))$\;
    $vr = sign\_change(P(x+r))$\;
    \If{$vl - vr = 1$}{
      $ret.append([l,r])$\;
    }
    \ElseIf{$vl - vr > 1$ and $r - l \geq MINIMAL\_RANGE $}{
      $mid = l + (r-l)/2$\;
      $search.append(mid, r)$\;
      $search.append(l, mid)$\;
    }
  }

  \Return{ret}\;
\caption{Real-root isolation based on Budan's Theorem}
\end{algorithm}

In Algorithm~\ref{alg3}, $Upper(P)$ returns the upper bound of real roots of
$P$. This project uses Lagrange's bound\cite{Lag} in this project. Assuming $P =
a_0 + a_1x + ... + a_nx^n$, Lagrange's bound is $max\{1,
\sum_{i=0}^{n-1}|\frac{a_i}{a_n}|\}$.





\subsection{Continued Fraction}

Let's first introduce some notations used in this algorithm. Let $M(x)$
represent a Mobius transformation, which maps $x$ to $\frac{ax+b}{cx+d}$. So
that the number of positive roots of $P(M(x))$ equals the number of roots in
interval $(\frac{b}{d}, \frac{a}{c}]$ of $P$. Denote the sign changes of 
coefficients of $P$ as $s$.

We use $\{a,b,c,d,p, s\}$ to represent an interval. Where, $ad-bc \neq 0$ and the
roots of original polynomial $P$ in interval $(\frac{b}{d},\frac{a}{c}]$ are
images of positive roots of $p$.

Continued fraction method can be formalized as Algorithm~\ref{alg4}. 

\begin{algorithm}[H]
\label{alg4}
\SetAlgoLined
  \SetKwInOut{Input}{input}
  \SetKwInOut{Output}{output}

  \Input{A square-free polynomial with no zero root $P$} 
  \Output{List of intervals contains only one positive real root}

  $s = sign\_change(P)$\;
  \If{$s=0$}{ 
    \Return{[]};
  }
  \ElseIf{$s=1$}{
    \Return{$[(0,\infty)]$};
  }

  $ret = []$\;
  $intervals = [\{1,0,0,1,P,s\}]$\;

  \While{$intervals$ not empty}{
    $\{a,b,c,d,p,s\} = pop(intervals)$\;
    $p' = p(x+1)$\;

    \If{$p'(0)=0$}{
      $ret.append([\frac{b}{d},\frac{b}{d}])$\;
      $p' = p'/x$\;
    }

    $s' = sign\_change(p')$\;
      $intervals.append(\{a, a+b, c, c+d, p', s' \})$\;

    \If{$s-s' = 1$}{
      $ret.append([\frac{a}{c},\frac{b}{d}])$\;
    }
    \ElseIf{$s-s' > 1$} {
      % add A(b/(1 + x)),  M(b/(1 + x)) to L
      $p'' = p(b/(1+x))$\;
      $intervals.append(\{b, a+b, d, c+d, p'', sign\_change(p'') \})$\;

    }
  }

  \Return{ret}\;
\caption{Real-root isolation based on Continued Fraction}
\end{algorithm}

Algorithm~\ref{alg4} can only returns positive real roots, therefore,
Algorithm~\ref{alg4} will be applied both on $P(x)$ and $P(-x)$. Zero roots
should be handled before using Algorithm~\ref{alg4}. 
