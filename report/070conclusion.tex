\section{Conclusion} \label{conclusion}

In this project, we implement a real-roots isolation program based on
Budan’s theorem and continued fraction. Besides that, we also use several
methods to optimize the program, like Taylor Expansion for polynomial shift and
interval arithmetic for error control.

Then we compare the running time of these two algorithms. It is noticeable that
when there are conjugate complex roots, Budan’s theorem will take longer. And
continued fraction methods perform worse when polynomials have several large
real roots due to slow shifts of polynomials. This can be optimized in the
future.

With the application of interval arithmetic, our program becomes more robust to
the errors from inexact computation and representation, although it is not
perfect for all of the cases. We analyze the conditions that might
cause failure of the program. We find that after square-free decomposition,
every factors has error in express of coefficients. It is hard to isolate close
real roots with this expression since small change of coefficients might causes
large change the property of these roots.


