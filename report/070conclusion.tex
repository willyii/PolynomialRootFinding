\section{Conclusion}

In this project, I tried to implemented real-roots isolation program based on
Budan's Theorem and Continued Fraction methods. Besides that, I also tried
several methods to optimize the program, like Tylor Expansion for polynomial
shift and interval arithmetic for error control. 

Then we compare the running time of both methods. In general, continued fraction
methods needs lesser computation time for polynomials with arbitrary
coefficients. But, for the polynomials with large roots, continued fraction
might take longer, since slow shift of polynomial and this can be optimized in
the future. 

With the interval arithmetic, I tried to analyze the conditions that might
cause failure of the program. And found that the tolerance of intervals might be
enlarged in the process of $remainder$ and $GCD$ computation. If the magnitude
of coefficient is large and the roots are close, program might failed.
