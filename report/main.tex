%%
%% This is file `sample-acmsmall.tex',
%% generated with the docstrip utility.
%%
%% The original source files were:
%%
%% samples.dtx  (with options: `acmsmall')
%% 
%% IMPORTANT NOTICE:
%% 
%% For the copyright see the source file.
%% 
%% Any modified versions of this file must be renamed
%% with new filenames distinct from sample-acmsmall.tex.
%% 
%% For distribution of the original source see the terms
%% for copying and modification in the file samples.dtx.
%% 
%% This generated file may be distributed as long as the
%% original source files, as listed above, are part of the
%% same distribution. (Tdrhe sources need not necessarily be
%% in the same archive or directory.)
%%
%% The first command in your LaTeX source must be the \documentclass command.
\documentclass[acmsmall,nonacm]{acmart}
%% NOTE that a single column version is required for 
%% submission and peer review. This can be done by changing
%% the \doucmentclass[...]{acmart} in this template to 
%% \documentclass[manuscript,screen]{acmart}
%% 
\usepackage{tcolorbox}
\usepackage{moresize}
\usepackage[ruled,vlined]{algorithm2e}

\newcommand{\M}{\textbf{M}}
\newcommand{\D}{\textbf{D}}

%% To ensure 100% compatibility, please check the white list of
%% approved LaTeX packages to be used with the Master Article Template at
%% https://www.acm.org/publications/taps/whitelist-of-latex-packages 
%% before creating your document. The white list page provides 
%% information on how to submit additional LaTeX packages for 
%% review and adoption.
%% Fonts used in the template cannot be substituted; margin 
%% adjustments are not allowed.
%%
%% \BibTeX command to typeset BibTeX logo in the docs
\AtBeginDocument{%
  \providecommand\BibTeX{{%
    \normalfont B\kern-0.5em{\scshape i\kern-0.25em b}\kern-0.8em\TeX}}}
\newcommand{\etal}{\textit{et al.}}

%% Rights management information.  This information is sent to you
%% when you complete the rights form.  These commands have SAMPLE
%% values in them; it is your responsibility as an author to replace
%% the commands and values with those provided to you when you
%% complete the rights form.
\setcopyright{acmcopyright}
\copyrightyear{2018}
\acmYear{2018}
\acmDOI{10.1145/1122445.1122456}


%%
%% These commands are for a JOURNAL article.
\acmJournal{JACM}
\acmVolume{37}
\acmNumber{4}
\acmMonth{8}

%%
%% Submission ID.
%% Use this when submitting an article to a sponsored event. You'll
%% receive a unique submission ID from the organizers
%% of the event, and this ID should be used as the parameter to this command.
%%\acmSubmissionID{123-A56-BU3}

%%
%% The majority of ACM publications use numbered citations and
%% references.  The command \citestyle{authoryear} switches to the
%% "author year" style.
%%
%% If you are preparing content for an event
%% sponsored by ACM SIGGRAPH, you must use the "author year" style of
%% citations and references.
%% Uncommenting
%% the next command will enable that style.
%%\citestyle{acmauthoryear}

%%
%% end of the preamble, start of the body of the document source.
\begin{document}

%%
%% The "title" command has an optional parameter,
%% allowing the author to define a "short title" to be used in page headers.
\title{Real-Root Isolation of Polynomials}

%%
%% The "author" command and its associated commands are used to define
%% the authors and their affiliations.
%% Of note is the shared affiliation of the first two authors, and the
%% "authornote" and "authornotemark" commands
%% used to denote shared contribution to the research.
\author{Xinlong Yi}
\email{xyi007@ucr.edu}
\affiliation{%
  \institution{\\Computer Science and Engineering Department, University of California Riverside}
  \streetaddress{900 University Ave}
  \city{Riverside}
  \state{California}
  \country{USA}
  \postcode{92507}
}


%%
%% By default, the full list of authors will be used in the page
%% headers. Often, this list is too long, and will overlap
%% other information printed in the page headers. This command allows
%% the author to define a more concise list
%% of authors' names for this purpose.
% \renewcommand{\shortauthors}{Trovato and Tobin, et al.}

%%
%% The abstract is a short summary of the work to be presented in the
%% article.

%%
%% The code below is generated by the tool at http://dl.acm.org/ccs.cfm.
%% Please copy and paste the code instead of the example below.
%%
% \begin{CCSXML}
% <ccs2012>
%  <concept>
%   <concept_id>10010520.10010553.10010562</concept_id>
%   <concept_desc>Computer systems organization~Embedded systems</concept_desc>
%   <concept_significance>500</concept_significance>
%  </concept>
%  <concept>
%   <concept_id>10010520.10010575.10010755</concept_id>
%   <concept_desc>Computer systems organization~Redundancy</concept_desc>
%   <concept_significance>300</concept_significance>
%  </concept>
%  <concept>
%   <concept_id>10010520.10010553.10010554</concept_id>
%   <concept_desc>Computer systems organization~Robotics</concept_desc>
%   <concept_significance>100</concept_significance>
%  </concept>
%  <concept>
%   <concept_id>10003033.10003083.10003095</concept_id>
%   <concept_desc>Networks~Network reliability</concept_desc>
%   <concept_significance>100</concept_significance>
%  </concept>
% </ccs2012>
% \end{CCSXML}

% \ccsdesc[500]{Computer systems organization~Embedded systems}
% \ccsdesc[300]{Computer systems organization~Redundancy}
% \ccsdesc{Computer systems organization~Robotics}
% \ccsdesc[100]{Networks~Network reliability}

%%
%% Keywords. The author(s) should pick words that accurately describe
%% the work being presented. Separate the keywords with commas.
% \keywords{datasets, neural networks, gaze detection, text tagging}


%%
%% This command processes the author and affiliation and title
%% information and builds the first part of the formatted document.
\maketitle
% Presentation link: \url{https://docs.google.com/presentation/d/1M79AJubJ6nv7tiNiWuUS_wnHdhcYuvaeLgdKnoC8FTw/edit?usp=sharing}

% \newpage

\section{Abstract}

Computing the real roots of univariate polynomial is one of fundamental tasks in
numeric and computer algebra. Usual root-finding algorithms for computing the
real roots of a polynomial may produce some real roots, but cannot generally
certify having found all real roots.

In this project, we implemented a basic real roots isolation program based on
Budan's Theorem and Continued Fraction Method. Both methods are developed from 
Descartes' rule of signs. Then we compare the running time of these two methods
and get the conclusion that TODO.


\section{Introduction}

One of the most fundamental scientific computations is computing the real roots
of polynomials. For the polynomials with low order, like quadratic or cubic, we
can use formulas to get roots directly. However, according to the Abel–Ruffini
theorem\cite{Abel-Ruffini}, there is no solution in radicals to general
polynomial equations with five degrees or higher with arbitrary coefficients.
Therefore, general root-finding algorithms are needed for arbitrary polynomials.

However, the usual root-finding algorithms, like Newton Method, cannot generally
certify having found all real roots. Especially, if such algorithms does not
find any root, one cannot know if there are real roots or not. In order to get
all real roots, real-root isolation is useful. Real-root isolation can generate
intervals, which contain only one real root of the polynomial, so that no real
root will be missed.

In this project, I implemented two real-root isolation algorithms based on
Budan's Theorem\cite{Budan} and Continued Fraction. After basic implemented, I optimized the
program with better data structure and interval arithmetic. After that, I compare
the running time of two algorithms and analyze in what condition these methods will fail. 

The organization of this report is as follows. Section 3 will
introduce the methodologies that used to implement the real-root isolation.
Section 4 has the way that I implemented this project and what I tried to
optimize it. And running time comparison and how error propagated  
will be discussed in Section 5.


%\section{Literature Review}

\section{Methodology}

In this section, 

\section{Implementation}

\section{Results and analysis} \label{analysis}

\subsection{Running Time}

With above implementation and operation, we test the running time of these two
methods. Results can be found in table~\ref{tb2}. Since we do not pay attention
to the polynomials with degree more than 6, we only test the polynomials with
degree 6 in this section. These test polynomials are generated with random
coefficients from $-c$ to $c$. 

\begin{center}
\label{tb2}
\begin{tabular}{ |c|c|c| } 
 \hline

 $c$  & Budan's Theorem & Continued Fraction\\ 

 \hline
 10   & 195 $us$  & 26 $us$\\ 
 20   & 240 $us$  & 22 $us$\\ 
 50   & 243 $us$  & 32 $us$\\ 
 100   & 257 $us$  & 32 $us$\\ 
 1000   & 259 $us$  & 32 $us$\\ 
 \hline
\end{tabular}
\end{center}

From table~\ref{tb2}, we can see that Budan's theorem method takes longer than
the continued fraction method. This might be because in Budan’s theorem, we need
to get the upper bound of root $up\_b$ first and get $P(x+up\_b)$ to check how
many positive real roots. However in continued fraction, sign variance of $P$
can tell us the information about positive roots directly. 

Furthermore, in Budan's theorem method, sometimes the search range needs to be
small enough to figure out whether there are roots or not. Take $P(x) = x^4 +1 $
as an example. In Budan's theorem method with search range $[0-\epsilon,
0+\epsilon]$, $v_{0+\epsilon}(P) - v_{0-\epsilon}(P) = 4$ is always valid, no
matter how small the $\epsilon$ is. Therefore, the bisection process will
terminate only when $2\epsilon < MINIMAL\_RANGE$. However, the continued
fraction method will check zero root first, then find no positive root since
$v_0(P(x)) = 0$ and no negative roots since $v_0(P(-x))=0$. That's the reason
why the Budan’s theorem method takes longer than the continued fraction method.

Above polynomials are generated with random coefficients, which means it cannot
guarantee that those polynomials have real roots. In order to figure out the
relationship between the running time with magnitude of roots, we designed
following experiments. Test polynomials have 6 roots and every root has $p =
0.5$ probability same as the previous one. Distinct roots are generated randomly
in range $[-max\_root, max\_root]$. Degree of test polynomials is 6. Running
times are shown in table~\ref{tb3}.


\begin{center}
\label{tb3}
\begin{tabular}{ |c|c|c| } 
 \hline

 $max\_root$  & Budan's Theorem & Continued Fraction\\ 

 \hline
 1   & 23 $us$  & 23 $us$\\ 
 10   & 28 $us$  & 25 $us$\\ 
 50   & 34 $us$  & 36 $us$\\ 
 100   & 36 $us$  & 51 $us$\\ 
 200   & 39 $us$  & 86 $us$\\ 
 500   & 42 $us$  & 180 $us$\\ 
 1000   & 47 $us$  & 300 $us$\\ 
 \hline
\end{tabular}
\end{center}

First, we can observe that when roots go larger the continued fraction method
takes longer than the Budan's theorem method. It is noticeable that $p'=p(x+1)$
step in Algorithm~\ref{alg4} only shifts polynomial by $1$ each time. Therefore,
when roots are large, it takes longer for the continued fraction method to shift
the polynomial to the place around roots. This place needs to be optimized in
the future.  Shifting can be flexible according to the lower bound of roots of
polynomials.

Furthermore, we can also notice that the Budan’s theorem method runs faster than the 
previous one while the continued fraction method runs slower. This mainly
because these polynomials are guaranteed to have 6 real roots. With 6 real
roots, there are no conjugate complex roots, like $x^4-1$. This condition makes
Budan’s theorem method terminate earlier. Besides that, some factors of original
polynomial might only have degree of 2 or 1. These factors can be solved very
quickly. However, for the continued fraction method, it takes longer to shift
polynomials to the place around roots.

\subsection{Error Analysis}

Although this implementation can work correctly in most cases, there are some
conditions that could lead to failure.

Since both methods need to work on polynomials with no repeat roots, square-free
decomposition becomes the most important step in the program. However, Yun's
algorithm requires exact divisions and $GCD$ success, which is hard to achieve
in floating point computation.

As mentioned before, we introduce interval arithmetic to handle the errors from
inexact computation and keep track of errors. We find that errors are related to
the magnitude of coefficients. Errors might be enlarged during the process of
$GCD$ and long division. Let's simulate the first step of $GCD(P,P')$, which is
$rem(P,P')$ where $P' = \frac{dP}{dx}$.

Given a polynomial $P=a_0+a_1*x + a_2*x^2...+a_n*x^n$, and its derivative
$P'=a_1+2*a_2*x...+n*a_n*x^{n-1}$. We assume all of coefficients of $P$ and $P'$
can be represented by computer perfectly. Which means the width of interval
$a_i$ is zero.

Then the processes to calculate $rem(P,P')$ are:

\begin{align*}
  div1 &= 1/n \\ 
  rem1 &= P - div_1 * x * P' =  a_0 + \frac{n-1}{n}a_1*x+ \frac{n-2}{n}a_2*x^2....
\frac{1}{n}*a_{n-1}x^{n-1}\\ 
       &= \sum_{i=0}^{n-1} \frac{n-i}{n} a_i x^i \\
  div2 &= \frac{a_{n-1}}{a_n} * \frac{1}{n^2} \\ 
       &= div_1 * \frac{a_{n-1}}{na_n}\\
  rem2 &= rem1 - div2 * P' = (a_0 - div2*a_1) + (\frac{n-1}{n}*a_1 - div2 *
  2*a_2)+ ... \\
      &= \sum_{i=0}^{n-2} (\frac{n-i}{n}*a_i -
      \frac{a_{n-1}*a_{i+1}}{n^2*a_n}(i+1)) x^i \\
      &= \sum_{i=0}^{n-2} ((1-div1*i)*a_i -
       div2 * a_{i+1}(i+1)) x^i
\end{align*} 

Since $rem2$ only has degree $n-2$ which is one less than $P'$, it is the answer
of $rem(P,P')$. Let's assume errors appear in computation of $div1$. Which means 
$div1 = \frac{1}{n}\pm \epsilon$. Then we assume $\xi = \frac{a_{n-1}}{na_n}$
is exact but very large, like $1e^{20}$. Then the $div2$ becomes
$\frac{a_{n-1}}{a_n*n^2} \pm \epsilon * \xi$. Since $\xi$ is very large, the
error of $div2$ is not ignorable. And it will affect the coefficients of $rem2$,
which makes the $rem2$ not accurate.

Therefore, if the magnitude of coefficients is very large, there is potential for
this program to fail in $GCD$ and long division processes and lead to failure in
real-root isolation.


\section{Conclusion}

In this project, I tried to implemented real-roots isolation program based on
Budan's Theorem and Continued Fraction methods. Besides that, I also tried
several methods to optimize the program, like Tylor Expansion for polynomial
shift and interval arithmetic for error control. 

Then we compare the running time of both methods. In general, continued fraction
methods needs lesser computation time for polynomials with arbitrary
coefficients. But, for the polynomials with large roots, continued fraction
might take longer, since slow shift of polynomial and this can be optimized in
the future. 

With the interval arithmetic, I tried to analyze the conditions that might
cause failure of the program. And found that the tolerance of intervals might be
enlarged in the process of $remainder$ and $GCD$ computation. If the magnitude
of coefficient is large and the roots are close, program might failed.



%%
%% The acknowledgments section is defined using the "acks" environment
%% (and NOT an unnumbered section). This ensures the proper
%% identification of the section in the article metadata, and the
%% consistent spelling of the heading.
% \begin{acks}
% % To Robert, for the bagels and explaining CMYK and color spaces.
% \end{acks}

%%
%% The next two lines define the bibliography style to be used, and
%% the bibliography file.
\bibliographystyle{ACM-Reference-Format}
\bibliography{references}


%%
%% If your work has an appendix, this is the place to put it.
\appendix


\end{document}
\endinput
%%
%% End of file `sample-acmsmall.tex'.
