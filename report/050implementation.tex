\section{Implementation}

This section will introduce how I implement this project and what I tried to
optimize it.

\subsection{Polynomial} 

At first, I tried to use \textit{vector} from \textit{std} library to implement
polynomial class. But due to time consuming of dynamic memory allocation
process. I changed it to static process.  

I use template class with an integer template $n$ to represents the
maximum possible degree of polynomial. Which means although a polynomial class with
template $n$, it's highest degree could be $n1$ where $n1 \leq n$. Although in
this way, programmer need to handle the possible change over the template when
writing some operators, like $*,+$, program can get rid of dynamic memory allocation,
which can save a lot of running time.

\subsection{Replace $x$ with $x+a$}

Changing $x$ to $x+a$ in a polynomial is a very common operation in both
methods, since I need to use the coefficients of $P(x+a)$ and get the sign
variance. 

At first, I used very primitive idea for this process. Initialize with very
simple polynomial $x+ a$. Then multiply it by coefficients and increase its
order one by one. Add them together at last to get transformed polynomial.

In order to optimize the process that replace $x$ to $x+a$ in a polynomial, I applied Taylor
expansion. If I want to change $P(x)$ to $P(x+a)$, I can take Taylor expansion
at point $a$ with $\Delta x = x$. Then:

$$
P(x) = P(a) + P'(a) *x + ... + \frac{1}{n!}P^{(n)}(a)*x^n
$$

This process avoid multiplication of two polynomials, which needs $O(n^2)$
operations in my implementation. In order to make sure Taylor expansion can
improve the perform of program, I made some experiments to compare the
computation time of changing $x$ to $x+a$. Reuslts are in table~\ref{tb1}.
Coefficients in test polynomial are generated randomly from $-n$ to $n$ and
degree of test polynomial is $n-1$.

\begin{center}
\label{tb1}
\begin{tabular}{ |c|c|c| } 
 \hline

 $n$  & Tylor Expansion   & Original Method\\ 

 \hline
 10   & 12$us$  & 16 $us$\\ 
 20   & 42$us$  & 58 $us$\\ 
 50   & 250$us$  & 340 $us$\\ 
 100   & 986$us$  & 1345 $us$\\ 
 150   & 2169$us$  & 3082 $us$\\ 
 200   & 3801$us$  & 5784 $us$\\ 
 \hline
\end{tabular}
\end{center}

As shown above, the running time of polynomial shifting with Tylor Expansion is
about $25\%$  faster than the primitive one in my implementation.

Appling Fast Fourier Transform can speed up polynomial multiplication process.
Maybe I will implement it in the future.

\subsection{Approximate GCD}

With above implementation, it cannot perform square-free decomposition
correctly all the time. The main reason behind this is the representation of float in
computer and computation error. 

In order to solve such error problem, I tried to use tolerance. If a number
smaller than the tolerance, it will be treated as zero. At first I used fixed
tolerance, which cannot perform well as the magnitude of largest coefficient
grows.

Then, I tried to make the original polynomial
\textit{regular}\cite{Approximate}, which means make the leading coefficient of
polynomial to $O(1)$ and remaining coefficients either $O(1)$ or $0$ by scaling
transformation $P\rightarrow \xi P $ and $x \rightarrow  \eta x$. The notation $O(c)$
not same as the Landau’s symbol and it means a number of approximately the same
magnitude as $c$.

However, changing of coefficients and roots will introduce extra computation.
Besides that, in some cases that have several extremely large roots, changing
coefficients to make them $O(1)$ will make some coefficients very small and loose the
percision. 

\subsection{Interval arithmetic}

With above problem, I discard the methods introduced by
Matu-Tarow\cite{Approximate} and applied \textit{interval arithmetic} to the
project. Every exact number in project has been changed to a interval that
represents a set of possible values. If zero in such interval, this number will
be seen as zero.

With interval arithmetic, the computation has been doubled since every
computation need to computation both ends of interval. However, this
implementation can handle the errors from float computation and data measuring.

By using interval arithmetic, I observed how error propagates during the GCD
process. It's close related the magnitude of coefficient of polynomial. This
will be discussed in Analysis~\ref{Analysis} section. 

