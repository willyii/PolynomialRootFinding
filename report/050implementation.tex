\section{Implementation}

This section will introduce how I implement this project and what I tried to
optimize it.

\subsection{Polynomial} 

At first, I tried to use \textit{vector} from \textit{std} library to implement
polynomial class. But due to time consuming of dynamic memory allocation
process. I changed it to static process.  

I use template class with an integer template $n$ to represents the
maximum possible degree of polynomial. Which means although a polynomial class with
template $n$, it's highest degree could be $n1$ where $n1 \leq n$. Although in
this way, programmer need to handle the possible change over the template when
writing some operators, like $*,+$, program can get rid of dynamic memory allocation,
which can save a lot of running time.

\subsection{Replace $x$ with $x+a$}

In order to optimize the process that replace $x$ to $x+a$ in a polynomial, I applied Taylor
expansion. If I want to change $P(x)$ to $P(x+a)$, I can take Taylor expansion
at point $a$ with $\Delta x = x$. Then:

$$
P(x) = P(a) + P'(a) *x + ... + \frac{1}{n!}P^{(n)}(a)*x^n
$$

% TODO:Experiment show optimization 

\subsection{Approximate GCD}

With above implementation, it cannot perform square-free decomposition
correctly. The main reason behind this is the representation of float in
computer and computation error. 

In order to solve such error problem, I tried to use tolerance. If a number
smaller than the tolerance, it will be treated as zero. At first I used fixed
tolerance, which cannot perform well as the magnitude of largest coefficient
grows.

Then, I tried to make the original polynomial
\textit{regular}\cite{Approximate}, which means make the leading coefficient of
polynomial to $O(1)$ and remaining coefficients either $O(1)$ or $0$ by scaling
transformation $P\rightarrow \xi P $ and $x \rightarrow  \eta x$. The notation $O(c)$
not same as the Landau’s symbol and it means a number of approximately the same
magnitude as $c$.

% TODO: correct this paragraph
However, changing of coefficients and roots will introduce extra computation.
Besides that, in some cases that have several extremely large roots, changing
coefficients to make them $O(1)$ will make some coefficients extremely small and
loss some accuracy even miss real roots.

\subsection{Interval arithmetic}

With above problem, I discard the methods introduced by
Matu-Tarow\cite{Approximate} and applied \textit{interval arithmetic} to the
project. Every exact number in project has been changed to a interval that
represents a set of possible values. If zero in such interval, this number will
be seen as zero.

With interval arithmetic, the computation has been doubled since every
computation need to computation both ends of interval. However, this
implementation can handle the errors from float computation and data measuring.

\subsection{Analysis}

% TODO: close roots
% TODO: Large roots
% TODO: Small roots
% TODO: Falied condition 
% TODO: Percision
% TODO: Running Time
