\section{Introduction}

One of the important scientific computing problems is computing the real roots
of polynomials. It has wide applications in computer graphics. For the
polynomials with low order, like quadratic or cubic, we can use formulas to get
roots directly. However, according to the Abel–Ruffini theorem, there is no
solution in radicals to general polynomial equations of five degrees or higher
with arbitrary coefficients\cite{Abel-Ruffini}.

There are a lot of algorithms to find the real roots of polynomials. Although
most root-finding algorithms, Newton’s method for example, may produce some real
roots, the convergence of algorithms are not guaranteed. Furthermore, these
methods cannot generally certify having found all real roots. Which means if
such methods do not find any root, one cannot know if there are real roots or
not. Moreover, there are some cases where one does not need the exact roots of
polynomial equations. Take ray tracing as an example. When computing the
distance from intersection point to ray’s end point, the object with smallest
distance is important to us, instead of exact distance to point of intersection.
In these cases, computing exact roots is not very essential.

Real-root isolation is an appropriate method for the above problems. It will
generate a sequence of disjoint intervals. Each interval contains only one real
root of the polynomial. And combining these intervals together, all real roots
could be found. Besides that, isolating the roots instead of computing them out
might speed up the cases that do not need exact roots, as mentioned above. 

This project aims to implement a real-root isolation program based on Budan’s
theorem and continued fraction. These two methods are based on Descartes’ rule
of signs, which describes how to get information on the number of positive real
roots of a polynomial and first introduced by René Descartes\cite{rule_of_sign}. Budan’s theorem,
developed from Descartes’ rule of signs, provides the methods to bound the
number of real roots of a polynomial in an interval\cite{Budan}. Vincent introduced the
continued fraction method in his work in 1834\cite{Vincent}. Both methods work only on
square-free polynomials. Therefore, this project takes Yun’s algorithm to
perform square free decomposition\cite{Yuns}.

The organization of this report is as follows. Section \ref{methods} will introduce the
theory basis of this project. Section \ref{implementation} describes how the program implemented
and what have been tried to optimize. Running time comparison and analysis of
how error propagation through computation will be discussed in Section
\ref{analysis}. Finally, a conclusion is drawn in Section \ref{conclusion}.
