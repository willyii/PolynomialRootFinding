\section{Introduction}

One of the most fundamental scientific computation is to computing the real
roots of polynomials. For the polynomials with low order, like quadratic or
cubic, we can using formula to get roots directly. However, according to the 
Abel–Ruffini theorem\cite{Abel-Ruffini}, there is no solution in radicals to general 
polynomial equations of degree of five or higher with arbitrary coefficients. 
Therefore, general root-finding algorithms are needed for general polynomials.

However, the usual root-finding algorithms, like Newton Method, cannot generally
certify having found all real roots. Especially, if such algorithms does not
find any root, one cannot know if there is real roots or not. In order to get
all real roots, real-root isolation is useful. Real-root isolation can generate
intervals, which contain only one real root of the polynomial, so that no real
root will be missed.

In this project, I implemented two real-root isolation algorithms based on
Budan's Theorem and Continued Fraction. After basic implemented, I optimized the
program with better data structure and interval arithmetic. Then I tried to
analysis the percision of this program. After that, I compare the running time
of two algorithms and analysis in what condition these methods will fail. 

The origanization of this project is as follows. In Section 2, I will review the
development of methods that used in this project. In Section 3, I will
introduced the methodologies that used to implemente the real-root isolation.
Section 4 has the way that I implemented this project and what I tried to
optimize it. And the analysis of error and percision will be described in
Section 5.
