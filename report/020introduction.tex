\section{Introduction}

One of the most fundamental scientific computations is computing the real roots
of polynomials. For the polynomials with low order, like quadratic or cubic, we
can use formulas to get roots directly. However, according to the Abel–Ruffini
theorem\cite{Abel-Ruffini}, there is no solution in radicals to general
polynomial equations with five degrees or higher with arbitrary coefficients.
Therefore, general root-finding algorithms are needed for arbitrary polynomials.

However, the usual root-finding algorithms, like Newton Method, cannot generally
certify having found all real roots. Especially, if such algorithms does not
find any root, one cannot know if there are real roots or not. In order to get
all real roots, real-root isolation is useful. Real-root isolation can generate
intervals, which contain only one real root of the polynomial, so that no real
root will be missed.

In this project, I implemented two real-root isolation algorithms based on
Budan's Theorem\cite{Budan} and Continued Fraction. After basic implemented, I optimized the
program with better data structure and interval arithmetic. After that, I compare
the running time of two algorithms and analyze in what condition these methods will fail. 

The organization of this report is as follows. Section 3 will
introduce the methodologies that used to implement the real-root isolation.
Section 4 has the way that I implemented this project and what I tried to
optimize it. And running time comparison and how error propagated  
will be discussed in Section 5.

