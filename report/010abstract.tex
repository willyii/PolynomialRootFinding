\section{Abstract}

Finding real roots of polynomials is a fundamental problem in scientific
computing. This project aims to implement a real-root isolation program based on
Budan’s Theorem and Continued Fraction. Both methods are developed from
Descartes’ rule of signs and work only on square-free polynomials. Therefore,
square-free decomposition is the very first step to isolate the roots of
polynomials. This project applies Yun’s algorithm to make the polynomials have
no repeated roots. And in order to handle errors that come from floating point
computation, interval arithmetic is introduced to replace exact numbers. Besides
that, many methods are used to optimize the program and running time of these
two algorithms are compared. This report also discusses how errors propagated
through computation and how it affects the success of the program.
